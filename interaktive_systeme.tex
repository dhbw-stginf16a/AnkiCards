% -*- coding: utf-8 -*-
\documentclass[12pt]{article}
\special{papersize=3in,5in}
\usepackage[utf8]{inputenc}
\usepackage{amssymb,amsmath}
\pagestyle{empty}
\setlength{\parindent}{0in}
\begin{document}

\begin{note}
    \xplain{Welche eigenen Erfahrungen haben Sie schon mit besonders schlecht zu benutzenden Anwendungen gemacht? Was hat Sie dabei besonders gestört? War Ihnen das bewusst? Was können Sie konkret kritisieren? Fallen Ihnen Verbesserungen ein?}
    \begin{field}
        \begin{itemize}
            \item VVS Karten Automat -> Touch Screen nicht treffsicher
            \item Abkürzungen auf Tasten und Anzeigen für Laien nicht verständlich
            \item Telefon; Adressbuchaufruf
            \item Nicht konsistente Zurück/Undo-Funktion bei diversen Apps oder Handy-Menüs
            \item Fernbedienungen; Tasten zu weit auseinander, Videotext-Steuerung komliziert
        \end{itemize}
    \end{field}
\end{note}

\begin{note}
    \xplain{Wie unterscheiden sich Mensch und Computer in ihren Fähigkeiten zur Kommunikation? Welche Folgen hat das für die Mensch-Computer-Interaktion (MCI)?}
    \begin{field}
        EVA (Eingabe -> Verarbeitung -> Ausgabe) bei Mensch und Maschine. \\
        Unterschiede in:
        \begin{itemize}
            % Mehr Infos auf der Slide zu den einzelnen Punkten; Gülzow (5)
            \item Medien für "Aufnahme" und "Ausgabe" von Daten unterschiedlich
            \item Vorhandene und erforderliche "Wissen", um Daten zu interpretieren und in Informationen umzusetzen
            \item Art und Weise, wie auf Informationen reagiert werden kann
        \end{itemize}
    \end{field}
\end{note}

\begin{note}
    \xplain{Optimieren Sie eine Mobiltelefontastatur: Nach der Nummerneingabe muss die Taste “Abheben” zum Wählen gedrückt werden. Wieviel bringt eine Verkürzung des mittleren Abstands zwischen dem Ziffernblock und „Abheben“ von 3cm auf 1cm bei einer Tastengröße von 5x5mm?}
    \begin{field}
        Fitts' law: $t = a + b * \log_2 (\frac{d}{s} + 1)$ \\
        s == Size des Buttons; d = Distance Ursprung zum Zielbutton \\
        $t1 - t2 = b * 0.92$ -> Zeitersparnis von 90ms, bei b = 100ms
    \end{field}
\end{note}

\begin{note}
    \xplain{Bei welchen Aufgaben bei der Gestaltung von grafischen Benutzungsoberflächen sollten Gestaltgesetze beachtet werden?}
    \begin{field}
        \begin{itemize}
            \item Schaffung einer visuellen Ordnung
            \item Strukturierung
            \item Zusammengehörigkeit
            \item Lenkung der Aufmerksamkeit
        \end{itemize}
    \end{field}
\end{note}

\begin{note}
    \xplain{Welche Bedeutung hat der Kanal der taktilen Wahrnehmung in der Mensch-Computer-Interaktion?}
    \begin{field}
        \begin{itemize}
            \item Druck, Berührung, Virbration, Temperatur, Schmerz
            \item Mensch-Mensch Kommunikation:
                \begin{itemize}
                    \item Haut Erkundungsorgan der dinglichen Welt
                    \item sozialer Aspekt der Berührung
                \end{itemize}
            \item Mensch-Maschine Kommunikation:
                \begin{itemize}
                    \item Eingaben ohne hinzuschauen
                    \item Haptische Rückkopplung
                    \item Haptische Kommunikation; Tasten werden getastet
                \end{itemize}
            \item taktile Wahrnehmung - über die Haut
            \item kinästhetische Wahrnehmung - über Museln und Gelenke
            \item Propriozeption - über Sensoren im Körperinneren
            \item Haptische Wahrnehmung: taktil + kinästhetisch + propriozeptiv
        \end{itemize}
    \end{field}
\end{note}

\begin{note}
    \xplain{Wie können in der Mensch-Computer-Interaktion Missverständnisse vermieden werden?}
    \begin{field}
        \begin{itemize}
            \item Eindeutige Formulierungen
            \item Icons mit erklärendem Text
            \item Hilfefunktionen
            \item Klare Darstellung der erwarteten Eingabe
            \item Konsistentes Interaktionskonzept
        \end{itemize}
    \end{field}
\end{note}

\begin{note}
    \xplain{Nennen und begründen Sie einige Beispiele für Constraints und Affordances - gibt es kulturelle Unterschiede? Bei welchen Gegenständen würden sie sich Affordances oder Constraints wünschen?}
    \begin{field}
    Bsp. Contraints:
        \begin{itemize}
            \item Mauscursor kann nicht über den Bildschirmrand bewegt werden
            \item Ganschaltung (Rückwärtsgang)
            \item Puzzle, Kabelfarben
            \item Farbenwahlen für bestimmte Anwendungen
        \end{itemize}
    Bsp. Affordances:
        \begin{itemize}
            \item Touchscreen
            \item Feuermelder
        \end{itemize}
    \end{field}
\end{note}

\begin{note}
    \xplain{Welche Metapher würden Sie vorschlagen, um damit Leihwagen in einem Konzern zu verwalten?}
    \xplain{Z.B. stilisierte Autos, vielleicht ein Parkplatz. \\ Ggfs. an bekannte Systeme und Kontexte anpassen, sofern diese Schnittstellen aufweisen}
\end{note}

\begin{note}
    \xplain{Erklären Sie den Unterschied zwischen Konsistenz und Erwartungskonformität.}
    \xplain{Konsistenz: bezieht sich auf Einheitlichkeit innerhalb eines Programmes oder einer Programmfamilie. \\ Erwartungskonformität: bezieht zusätzlich die auf Erfahrung mit anderen
Systemen beruhenden Erwartungen des Anwenders mit ein }
\end{note}

\begin{note}
    \xplain{Welche Kriterien wiederholen sich in verschiedenen Richtlinien? Worauf deutet diese Redundanz hin?}
    \begin{field}
        \begin{itemize}
            \item vernünftige Fehlerbehandlung
            \item Rückkopplung
            \item Konsistenz
            \item Minimale Ablenkung der Aufmerksamkeit
        \end{itemize}
        Kommen öfter vor. Deuten auf eine hohe Relevanz.
    \end{field}
\end{note}

\begin{note}
    \xplain{Worauf sollte man achten, wenn man Icons für die Benutzerführung verwendet?}
    \begin{field} 
         - Funktion follows Form, da Anwender Icon ohne weiter Hilfe verstehen muss \\
         - Keine hohen Erwartungen an "allgemeine Verständlichkeit" von Bildern knüpfen
         \begin{itemize}
            \item Einheitlichkeit
            \item Standardisierung
            \item Übertreibung
            \item Kontextbezug
            \item Zielgruppenorientierung
            \item Farbwahl
        \end{itemize}
    \end{field}
\end{note}

\end{document}