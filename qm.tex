% -*- coding: utf-8 -*-
\documentclass[12pt]{article}
\special{papersize=3in,5in}
\usepackage[utf8]{inputenc}
\usepackage{amssymb,amsmath}
\pagestyle{empty}
\setlength{\parindent}{0in}
\begin{document}



\begin{note}
    \xplain{Qualität nach ISO 9000}
    \xplain{„Qualität ist der Grad, in dem ein Satz inhärenter Merkmale Anforderungen erfüllt.“}
\end{note}

\begin{note}
    \xplain{Was ist ein inhärentes Merkmal?}
    \begin{field}
        inhärentes ("innewohnendes") Merkmal eines Produkts, Prozesses oder Systems, kann;
        \begin{itemize}
            \item quantitativ (stetig -> (Körpergröße, Gewicht) oder diskret -> (Anzahl Kinder/Geschwister))
            \item qualitativ (ordinal -> (Dienstgrad, Akd. Titel) oder nominal -> (Blutgruppe, Haarfarbe))
        \end{itemize}
        Unterschiedliche Merkmale:
        \begin{itemize}
            \item physische, sensorische (z.B. Gechmack)
            \item verhaltensbezogen (z.B. Höflichkeit)
            \item zeitbezogen (z.B. Verfügbarkeit)
        \end{itemize}
        Qualitätsmanagement -> Produktprüfung && Produktrealisierungsprozess
    \end{field}
\end{note}

\begin{note}
    \xplain{Was ist ein Produkt?}
    \xplain{ - "Ergebnis einer Organisation, welche ohne Transaktion zwischen Organisation und Kunden erzeugt werden kann.“ (ISO 9000:2015)
             - Hardware, Software, Dienstleistungen}
\end{note}

\begin{note}
    \xplain{Was ist die besondere Stellung einer Dienstleistung?}
    \begin{field}
        \begin{itemize}
            \item Immaterialität. Weiterverkauf, Lagerung und Transport nicht möglich
            \item Verhaltensfehler = Produktfehler
            \item Produktion und Konsumierung ist gleichzeitig
            \item Direkter kontakt - Hersteller und Kunde
        \end{itemize}
    \end{field}
\end{note}

\begin{note}
    \xplain{Was ist sind Anforderungen?}
    \xplain{„Anforderungen sind Erfordernisse oder Erwartungen, die festlegt, üblicherweise vorausgesetzt oder verpflichtend sind.“ (ISO 9000:2015)}
\end{note}

\begin{note}
    \xplain{Nenne Qualitätsmerkmale}

    \begin{field}
        \begin{itemize}
            \item Funktionalität
            \item Zuverlässigkeit
            \item Benutzbarkeit
            \item Effizienz
            \item Wartungsfreundlichkeit
            \item Übertragbarkeit
        \end{itemize}
    \end{field}
\end{note}

\begin{note}
    \xplain{Qualitätsmerkmale einer Software}

    \begin{field}
        \begin{itemize}
            \item Software ist immateriell
            \item Viele Stakeholder, die die Qualität definieren
            \item Schnell ändernde Anforderungen - Veränderungen in der Umwelt, Anwenderanforderungen, Gesetze oder Marktsituationen
            \item Software Entwicklung ist ein kreativer Prozess, wird häufig vom Management unterschätzt (Whiskey-Syndrom) => erhöht die “Technischen Schulden”
        \end{itemize}
    \end{field}
\end{note}


\begin{note}
    \xplain{Modell von Kano - Kategorien}

    \begin{field}
        \begin{itemize}
            \item Begeisterungsanforderung: Unerwartete Mehrleistungen z.B. Fahrsicherheitstraining beim Kauf eines Sportwagens

            \item Leistungsanforderung: Vertraglich festgelegt z.B. Auto mit Klimaautomatik

            \item Basisanforderung: z.B. Auto mit vier Reifen

        \end{itemize}
    \end{field}
\end{note}

\begin{note}
    \xplain{Modell von Kano}

    \xplain{<img src="qm_kanto.png" />}
\end{note}

\begin{note}
    \xplain{Wie verhalten sich unzufriedene Kunden?}

    \begin{field}
        \begin{itemize}
            \item nur 4\% beschweren sich
            \item jeder unzufriedene Kunde spricht mit mindestens 10 Personen darüber
            \item 60\%-70\% der unzufriedenen kunden kommen nicht mehr oder wechseln zur Konkurrenz
            \item 98\% der Kunden akzeptieren Fehler, wenn diese umgehend beseitigt werden
        \end{itemize}
    \end{field}
\end{note}

\begin{note}
    \xplain{1000 Mitarbeiter, 2 Beschwerden pro Monat -> 24 Beschwerden pro Jahr}
    \xplain{Da sich nur 4\% der Leute beschwerden 50 Beschweren -> 600 pro Jahr}
\end{note}



\begin{note}
    \xplain{Was sind Taylors wissenschaftliche Regeln?}

    \begin{field}
        \begin{itemize}
            \item Auswahl der geeignetesten Person für die Arbeit

            \item Lehren der effizientesten Arbeitsmethoden und Bewegungen

            \item Höhere Leistung mit höherer Bezahlung belohnen
        \end{itemize}
    \end{field}
\end{note}

\begin{note}
    \xplain{Taylorismus - Nachteile}

    \begin{field}
        \begin{itemize}
            \item Trennung von Hand und Kopfarbeit
            \item Entfremdung zwischen Arbeiter und Produkt
            \item Demotivation
            \item Qualitätsmängel
            \item Verstärkung Qualitätskontrollen
        \end{itemize}
    \end{field}
\end{note}

\begin{note}
    \xplain{Vierzehn Managementprinzipien nach Deming (Demingsche Lehre)}

    \begin{field}
        \begin{itemize}
            \item Schafe den festen Willen zur ständigen Verbesserung im Unternehmen
            \item Schaffe ein Bewusstsein für Qualität
            \item Beseitige die Abhängigkeit von Vollkontrollen
            \item Richte dich nich tallein nach dem billigsten Angebot
            \item Verbressere ständig die Systeme
            \item Schaffe moderne Ausbildungsmethoden
            \item Sorge für richtiges Führungsverhalten
            \item Beseitige die Angst
            \item Beseitige Barrieren zwischen Geschäftsbereichen
            \item Setze postive Ziele statt negativer Kritik
            \item Betone die Qualität der Leistung, nicht die Quantität
            \item Ermögliche Stolz auf gute Arbeit
            \item Fördere Qualifikation und Weiterbildung
            \item Mache die ständige Verbesserung von Qualität und Produktivität zur Aufgabe der Unternehmensleistung
        \end{itemize}
    \end{field}
\end{note}

\begin{note}
    \xplain{Einflussfaktoren auf Qualität}

    \begin{field}
        \begin{itemize}
            \item Gesetzliche Auflagen: Produkthaftung, Arbeitsplatzsicherheit, Umweltschutz
            \item Kundenerwartungen: steigende Qualitätsansprüche, Wertewandel, geändertes Konsumverhalten
            \item Verschärfter Wettbewerb: Globalisierung, gesättigte Märkte -> Kostendruck, Überkapazitäten
            \item Unternehmen: Shareholder Value, zunehmende Produktkomplexität, schnellere Markteinführung neuer Produkte
        \end{itemize}
    \end{field}
\end{note}

\begin{note}
    \xplain{Spannungsfeld des Marktes}
    \xplain{<img src="qm_spannungsfeld.png" />}
\end{note}

\begin{note}
    \xplain{Auflösung des Spannungsfelds}
    \xplain{<img src="qm_aufloesung_spannung.png" />}
\end{note}

\begin{note}
    \xplain{Abgrenzung der Gesetze und P. 823 BGB}
    \xplain{<img src="qm_abgrenzung_gesetze.png" />}
\end{note}

\begin{note}
    \xplain{Produktsicherheitsgesetz; CE Kennzeichnung}
    \xplain{Kein Qualitäts- oder Gütesiegel, besagt, dass ein Produkt in jedem EU-Staat in Verkehr gebracht werden kann (weil es die entsprechenden Richtlinien einhält)}
\end{note}

\begin{note}
    \xplain{Produktsicherheitsgesetz; GS Kennzeichnung}
    \xplain{Gesetzlich geregeltes Gütesiegel für “Geprüfte Sicherheit”.}
\end{note}

\begin{note}
    \xplain{Was ist die Beweislastumkehr?}
    \begin{field}
        \begin{itemize}
            \item Gewährleistung: 6 Monaten nach Kauf muss Verkäufer beweisen. Danach Beweislastumkehr und der Verbraucher muss beweisen
            \item Produkthaftungsrecht: Beweislastumkehr, nicht der Geschädigte eine Herstellerpflichtverletzung zu beweisen auf der ein Schaden beruht, sondern der Hersteller
        \end{itemize}
    \end{field}
\end{note}

\begin{note}
    \xplain{Produzentenhaftung §823 BGB. Was muss ein Produzent tun um zu beweisen, dass er nicht schuldig ist:}

    \begin{field}
        \begin{itemize}
            \item Konstruktionspflichten
            \item Fabrikationspflichten
            \item Instruktionsplfichten
            \item Qualitätsmängel
            \item Produktbeobachtungspflichten
        \end{itemize}
    \end{field}
\end{note}

\begin{note}
    \xplain{Gründe für Produkthaftungsgesetz (Verschuldungsunabhängige Gefährdungshaftung)}

    \begin{field}
        \begin{itemize}
            \item Körperverletzung, Tod oder Beschädigung (irgend einer Sache)

            \item Produktfehler, der in der Sphäre des Herstellers entstanden ist

            \item Kausiltät zwischen Fehler und Schaden

            \item schuldhafte Sorgfaltspflichtverletzung
        \end{itemize}
    \end{field}
\end{note}

\begin{note}
    \xplain{Wann verwende ich Produzentenhaftung, wann verwendte ich Produkthaftung?}

    \begin{field}
        \begin{itemize}
            \item Produkthaftung: nur Verbraucher

            \item Produzentenhaftung: Verbraucher und Firmen
        \end{itemize}
    \end{field}
\end{note}
%ÎSO 9000

\begin{note}
    \xplain{Was sind die Einteilungen der Bereiche eines Unternehmens?}
    \xplain{<img src="qm_bereiche_unternehmen.png" />}
\end{note}

\begin{note}
    \xplain{Nenne drei verschiedene Prozessarten}

    \begin{field}
        \begin{itemize}
            \item Führungs-/Managementprozesse: z.B. Qualitäts-. Umwelt-, Sicherheitspolitik. Stellen generell Rahmenbedingungen für andere Prozeesse


            \item Kern-/Leistungsprozesse: z.B. Auftragsabwicklung und Produktentwicklung. Generieren Umsatz und Gewinn

            \item Unterstützende Prozesse: z.B. IT, Personal, Finanzen. Nicht direkt an Wertschöpfung beteiligt, aber notwendig für die anderen Prozesse
        \end{itemize}
    \end{field}
\end{note}


\begin{note}
    \xplain{Was sind Prozesseigenschaften}

    \begin{field}
        \begin{itemize}
            \item Abgegrenztheit - Beginn, Ende
            \item Detailierung - Zerlegung in Unterprozesse
            \item konkretes Ziel / Ergebins Ouput
        \end{itemize}
    \end{field}
\end{note}

\begin{note}
    \xplain{7M - Einflussfaktoren}

    \begin{field}
        \begin{itemize}
            \item Mensch
            \item Management
            \item Maschine
            \item Material
            \item Methode
            \item Messbarkeit
            \item Mitwelt (Umwelt)
        \end{itemize}
    \end{field}
\end{note}


\begin{note}
    \xplain{Was ist für ein dauerhaft erfolgreiches Unternehmen wichtig?}
    \xplain{Kontinuierliche Verbesserung und Innovation}
\end{note}

\begin{note}
    \xplain{Was ist Standardisieren?}
    \xplain{Die Erhaltung des Status quo}
\end{note}

\begin{note}
    \xplain{Grafische Darstellung von Standardisieren, KVP und Innovation}
    \xplain{<img src="qm_standard_innovation.png" />}
\end{note}

\begin{note}
    \xplain{Kaizen Werkzeug PDCA Kreis}
    \xplain{<img src="qm_pdca.png" />}
\end{note}

\begin{note}
    \xplain{Vorteile von PDCA}

    \begin{field}
        \begin{itemize}
            \item schnellerer Weg zu einer effektiven Problemlösung
            \item Arbeiten mit Daten und Fakten
            \item Ermittlung und Beseitigung des wahrscheinlichsten Fehlereinflusses
            \item Dokumentation des neuen Prozesses und Training
            \item Vorbeugemaßnahmen verhindern ein Wiederauftreten des Problems
        \end{itemize}
    \end{field}
\end{note}


\begin{note}
    \xplain{Geben Sie zwei Beispiele für einen PDCA}

    \begin{field}
        \begin{itemize}
            \item Six-Sigma, DMAIC-Zyklus
            \item QFD
            \item Benchmarking
        \end{itemize}
    \end{field}
\end{note}


\begin{note}
    \xplain{Folgen für mangelhafte Produktqualität}

    \begin{field}
        \begin{itemize}
            \item behördliche Weisungen und Sanktionen
            \item Gewährleistungsansprüche des Käufers gegen den Verkäufer
                \begin{itemize}
                    \item Zivilrecht - Vertragliche Haftung
                \end{itemize}
            \item Schadensersatzansprüche des Geschädigten auf außervertraglicher Produkthaftung
                \begin{itemize}
                    \item ProdHaftG und Paragraph 823 BGB
                \end{itemize}
            \item strafrechtliche Konsequenzen bei Körperverletzung / Tod
        \end{itemize}
    \end{field}
\end{note}

\begin{note}
    \xplain{zivilrechtliche Haftung}

    \begin{field}
        \begin{itemize}
            \item Gewährleistungspflicht
                \begin{itemize}
                    \item vertragliche Haftung des Unternehmens gegenüber dem Käufer, die aufgrund eines Kauf- oder Werkvertrags ensteht
                \end{itemize}
            \item außervertragliche Haftung
                \begin{itemize}
                    \item nach dem Produkthaftungsgesetzt (ProdHaftG) und dem bürgerlichen Gesetzbuch (BGB)
                \end{itemize}
            \item strafrechtliche Konsequenzen bei Körperverletzung / Tod
        \end{itemize}
    \end{field}
\end{note}


\end{document}
