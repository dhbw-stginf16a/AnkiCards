% -*- coding: utf-8 -*-
\documentclass[12pt]{article}
\special{papersize=3in,5in}
\usepackage[utf8]{inputenc}
\usepackage{amssymb,amsmath}
\pagestyle{empty}
\setlength{\parindent}{0in}
\begin{document}


\begin{note}
    \xplain{5 Sichtweisen der Qualität}

    \begin{field}
        \begin{itemize}
            \item Transzendent
            \item Produktbezogen
            \item Benutzerbezogen
            \item Prozessbezogen
            \item Kosten/Nutzen-bezogen
        \end{itemize}
    \end{field}
\end{note}

\begin{note}
    \xplain{Qualitätsichtweise; Transzendent}
    \xfield{Qualität absolut und universell, nicht präzise definierbar - muss erfahren werden. \\
            -\> nicht messbare Perfektion -\> kompromisslos hohe Ansprüche und Leistungen \\
            \\
            "Die Leistung des Fahrzeugs ist über jeden Zweifel erhaben"}
\end{note}


\begin{note}
    \xplain{Qualitätsichtweise; Produktbezogen}
    \xfield{Qualität präzise messbar, Qualitätsunterschide quantitativ vergleichbar. \\
            "Höchstgeschwindigkeit liegt bei 310 km/h"}
\end{note}


\begin{note}
    \xplain{Qualitätsichtweise; Benutzerbezogen}
    \xfield{Persönliche Produktnutzerbedürfnisse entscheiden über qualität. \\
            Am besten geeignetes / passendes Produkt \\
            \\
            "Weil es Menschen gibt, die das Offenfahren ... genießen"}
\end{note}


\begin{note}
    \xplain{Qualitätsichtweise; Prozessbesogen}
    \xfield{Qualität bei der Erstellung. Exakte Spezifikation \& Kontrolle. Genaues umsetzen und einhalten. \\
    \\
    "Unsere Modelle erfüllen mühelos die Euro-6-Abgasnorm"}
\end{note}


\begin{note}
    \xplain{Qualitätsichtweise; Kosten/Nutzen-bezogen}
    \xfield{Qualität als Funktion von Kosten und Nutzen. Verhältnis bestimmt über hochwertig oder minderwertig\\
            \\
            "Nicht nur günstig in der Anschaffung, sondern auch im Unterhalt ist das Fahrzeug unschlagbar in Preis und Leistung"}
\end{note}


\begin{note}
    \xplain{Qualität nach ISO 9000}
    \xplain{„Qualität ist der Grad, in dem ein Satz inhärenter Merkmale Anforderungen erfüllt.“}
\end{note}

\begin{note}
    \xplain{Was ist ein inhärentes Merkmal?}
    \begin{field}
        inhärentes ("innewohnendes") Merkmal eines Produkts, Prozesses oder Systems, kann;
        \begin{itemize}
            \item quantitativ (stetig -> (Körpergröße, Gewicht) oder diskret -> (Anzahl Kinder/Geschwister))
            \item qualitativ (ordinal -> (Dienstgrad, Akd. Titel) oder nominal -> (Blutgruppe, Haarfarbe))
        \end{itemize}
        Unterschiedliche Merkmale:
        \begin{itemize}
            \item physische, sensorische (z.B. Gechmack)
            \item verhaltensbezogen (z.B. Höflichkeit)
            \item zeitbezogen (z.B. Verfügbarkeit)
        \end{itemize}
        Qualitätsmanagement -> Produktprüfung && Produktrealisierungsprozess
    \end{field}
\end{note}

\begin{note}
    \xplain{Was ist ein Produkt?}
    \xplain{„Ein Produkt ist das Ergebnis einer Organisation, das ohne jegliche Transaktion zwischen Organisation und Kunden erzeugt werden kann.“ (ISO 9000:2015)}
\end{note}

\begin{note}
    \xplain{Was ist sind Anforderungen?}
    \xplain{„Anforderungen sind Erfordernisse oder Erwartungen, die festlegt, üblicherweise vorausgesetzt oder verpflichtend sind.“ (ISO 9000:2015)}
\end{note}

\begin{note}
    \xplain{Nenne Qualitätsmerkmale}

    \begin{field}
        \begin{itemize}
            \item Funktionalität
            \item Zuverlässigkeit
            \item Benutzbarkeit
            \item Effizienz
            \item Wartungsfreundlichkeit
            \item Übertragbarkeit
        \end{itemize}
    \end{field}
\end{note}

\begin{note}
    \xplain{Modell von Kano - Kategorien}

    \begin{field}
        \begin{itemize}
            \item Begeisterungsanforderung
            \item Leistungsanforderung
            \item Basisanforderung
        \end{itemize}
    \end{field}
\end{note}

\begin{note}
    \xplain{Modell von Kano}

    \xplain{<img src="qm_kanto.png" />}
\end{note}


\begin{note}
    \xplain{1000 Mitarbeiter, 2 Beschwerden pro Monat -> 24 Beschwerden pro Jahr}
    \xplain{Da sich nur 4\% der Leute beschwerden 50 Beschweren -> 600 pro Jahr}
\end{note}



\begin{note}
    \xplain{Taylorismus - Erklärung}

    \xfield{Mensch eine von aussen zu steuernde und kontrollierende „Kraftmaschine“. 
            Nicht die Zuverlässigkeit und Leistungsfähigkeit einer Maschine erreicht. \\
            Menschen handelt tendenziell rational und ist rein ökonomisch motiviert. }
\end{note}

\begin{note}
    \xplain{Taylorismus - Nachteile}

    \begin{field}
        \begin{itemize}
            \item Trennung von Hand und Kopfarbeit
            \item Entfremdung zwischen Arbeiter und Produkt
            \item Demotivation
            \item Qualitätsmängel
            \item Verstärkung Qualitätskontrollen
        \end{itemize}
    \end{field}
\end{note}

\begin{note}
    \xplain{Vierzehn Managementprinzipien nach Deming (Demingsche Lehre)}

    \begin{field}
        \begin{itemize}
            \item Schafe den festen Willen zur ständigen Verbesserung im Unternehmen
            \item Schaffe ein Bewusstsein für Qualität
            \item Beseitige die Abhängigkeit von Vollkontrollen
            \item Richte dich nich tallein nach dem billigsten Angebot
            \item Verbressere ständig die Systeme
            \item Schaffe moderne Ausbildungsmethoden
            \item Sorge für richtiges Führungsverhalten
            \item Beseitige die Angst
            \item Beseitige Barrieren zwischen Geschäftsbereichen
            \item Setze postive Ziele statt negativer Kritik
            \item Betone die Qualität der Leistung, nicht die Quantität
            \item Ermögliche Stolz auf gute Arbeit
            \item Fördere Qualifikation und Weiterbildung
            \item Mache die ständige Verbesserung von Qualität und Produktivität zur Aufgabe der Unternehmensleistung
        \end{itemize}
    \end{field}
\end{note}

\begin{note}
    \xplain{Qualität heute}

    \begin{field}
        \begin{itemize}
            \item Gesetzliche Auflagen: Produkthaftung, Arbeitsplatzsicherheit, Umweltschutz
            \item Kundenerwartungen: steigende Qualitätsansprüche, Wertewandel, geändertes Konsumverhalten
            \item Verschärfter Wettbewerb: Globalisierung, gesättigte Märkte -> Kostendruck, Überkapazitäten
            \item Unternehmen: Shareholder Value, zunehmende Produktkomplexität, schnellere Markteinführung neuer Produkte
        \end{itemize}
    \end{field}
\end{note}


\begin{note}
    \xplain{Folgen für mangelhafte Produktqualität}

    \begin{field}
        \begin{itemize}
            \item behördliche Weisungen und Sanktionen
            \item Gewährleistungsansprüche des Käufers gegen den Verkäufer
                \begin{itemize}
                    \item Zivilrecht - Vertragliche Haftung
                \end{itemize}
            \item Schadensersatzansprüche des Geschädigten auf außervertraglicher Produkthaftung
                \begin{itemize}
                    \item ProdHaftG und Paragraph 823 BGB
                \end{itemize}
            \item strafrechtliche Konsequenzen bei Körperverletzung / Tod
        \end{itemize}
    \end{field}
\end{note}

\begin{note}
    \xplain{zivilrechtliche Haftung}

    \begin{field}
        \begin{itemize}
            \item Gewährleistungspflicht
                \begin{itemize}
                    \item vertragliche Haftung des Unternehmens gegenüber dem Käufer, die aufgrund eines Kauf- oder Werkvertrags ensteht
                \end{itemize}
            \item außervertragliche Haftung
                \begin{itemize}
                    \item nach dem Produkthaftungsgesetzt (ProdHaftG) und dem bürgerlichen Gesetzbuch (BGB)
                \end{itemize}
            \item strafrechtliche Konsequenzen bei Körperverletzung / Tod
        \end{itemize}
    \end{field}
\end{note}


\end{document}
