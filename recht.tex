% -*- coding: utf-8 -*-
\documentclass[12pt]{article}
\special{papersize=3in,5in}
\usepackage[utf8]{inputenc}
\usepackage{amssymb,amsmath}
\pagestyle{empty}
\setlength{\parindent}{0in}
\begin{document}

\begin{note}
    \xplain{Was ist der unter Umständen bekanntere englischsprachige Begriff für geistiges Eigentum?}
    \xplain{Intellectual Property}
\end{note}

\begin{note}
    \xplain{Welche 4 Gesetze gehören zum gewerblichen Rechtsschutz im engeren Sinne?}
    \xplain{Marken-, Patent-, Design-, Gebrauchsmustergesetz}
\end{note}

\begin{note}
    \xplain{Was ist eine Marke?}
    \xplain{Kennzeichen zur Unterscheidung von Produkten von Unternehmen}
\end{note}

\begin{note}
    \xplain{Wie entstehen die gewerblich Schutzrechte, wie entsteht das Urheberrecht?}
    \xplain{Eintragung, Urheberrecht durch Schöpfung (automatisch)}
\end{note}

\begin{note}
    \xplain{Welche 3 Hauptansprüche hat ein Markeninhaber gegen den Verletzer einer Marke?}
    \xplain{Auskunft, Unterlassung und Schadensersatz}
\end{note}

\begin{note}
    \xplain{Wie lang kann eine Marke längstens bestehen?}
    \xplain{10 Jahre, verlängerung möglich. Unbegrenzt}
\end{note}

\begin{note}
    \xplain{Wo sind Erfindungen geregelt, die während der Arbeits enstehen und was ist dort geregelt?}
    \xplain{Arbeitnehmererfindergesetz: Erfindung gehört Arbeitgeber, Recht auf Namensnennung, Recht auf angemessenes Entgelt}
\end{note}

\begin{note}
    \xplain{Durch welches Gesetz ist Software explizit geschützt?}
    \xplain{Urheberrecht}
\end{note}

\begin{note}
    \xplain{Wer kann Urheber sein?}
    \xplain{Jede natürliche Person.}
\end{note}

\begin{note}
    \xplain{Wann erlischt das Urheberrecht?}
    \xplain{70 Jahre nach dem Tod des Urhebers.}
\end{note}

\begin{note}
    \xplain{Nach welchem Gesetz sind Personen auf Fotografien geschützt?}
    \xplain{Kunsturheberrecht}
\end{note}

\begin{note}
    \xplain{Wen schützt das UWG?}
    \xplain{Schutz der Mitbewerber und Verbraucher. (UWG = Gesetz gegen unlauteren Wettbewerb)}
\end{note}

\begin{note}
    \xplain{Ist vergleichende Werbung nach dem UWG zulässig?}
    \xplain{Ja, wenn wahr und nicht böse oder diskreditierend. (UWG = Gesetz gegen unlauteren Wettbewerb)}
\end{note}

\end{document}