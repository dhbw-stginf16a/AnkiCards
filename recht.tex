% -*- coding: utf-8 -*-
\documentclass[12pt]{article}
\special{papersize=3in,5in}
\usepackage[utf8]{inputenc}
\usepackage{amssymb,amsmath}
\pagestyle{empty}
\setlength{\parindent}{0in}
\begin{document}

\begin{note}
    \xplain{Was ist der unter Umständen bekanntere englischsprachige Begriff für geistiges Eigentum?}
    \xplain{Intellectual Property}
\end{note}

\begin{note}
    \xplain{Welche 4 Gesetze gehören zum gewerblichen Rechtsschutz im engeren Sinne?}
    \xplain{Marken-, Patent-, Design-, Gebrauchsmustergesetz}
\end{note}

\begin{note}
    \xplain{Was ist eine Marke?}
    \xplain{Kennzeichen zur Unterscheidung von Produkten von Unternehmen}
\end{note}

\begin{note}
    \xplain{Wie entstehen die gewerblich Schutzrechte, wie entsteht das Urheberrecht?}
    \xplain{Eintragung, Urheberrecht durch Schöpfung (automatisch)}
\end{note}

\begin{note}
    \xplain{Welche 3 Hauptansprüche hat ein Markeninhaber gegen den Verletzer einer Marke?}
    \xplain{Auskunft, Unterlassung und Schadensersatz}
\end{note}

\begin{note}
    \xplain{Wie lang kann eine Marke längstens bestehen?}
    \xplain{10 Jahre, verlängerung möglich. Unbegrenzt}
\end{note}

\begin{note}
    \xplain{Wo sind Erfindungen geregelt, die während der Arbeits enstehen und was ist dort geregelt?}
    \xplain{Arbeitnehmererfindergesetz: Erfindung gehört Arbeitgeber, Recht auf Namensnennung, Recht auf angemessenes Entgelt}
\end{note}

\begin{note}
    \xplain{Durch welches Gesetz ist Softwrae explizit geschützt?}
    \xplain{Urheberrecht}
\end{note}

\begin{note}
    \xplain{Wer kann Urheber sein?}
    \xplain{Jede natürliche Person.}
\end{note}

\begin{note}
    \xplain{Wann erlischt das Urheberrecht?}
    \xplain{70 Jahre nach dem Tod des Urhebers.}
\end{note}

\begin{note}
    \xplain{Nach welchem Gesetz sind Personen auf Fotografien geschützt?}
    \xplain{Kunsturheberrecht}
\end{note}

\begin{note}
    \xplain{Wen schützt das UWG?}
    \xplain{Schutz der Mitbewerber und Verbraucher. (UWG = Gesetz gegen unlauteren Wettbewerb)}
\end{note}

\begin{note}
    \xplain{Ist vergleichende Werbung nach dem UWG zulässig?}
    \xplain{Ja, wenn wahr und nicht böse oder diskreditierend. (UWG = Gesetz gegen unlauteren Wettbewerb)}
\end{note}

\begin{note}
    \xplain{Wie unterscheiden sich Mensch und Computer in ihren Fähigkeiten zur Kommunikation? Welche Folgen hat das für die Mensch-Computer-Interaktion (MCI)?}
    \xplain{Keine Ahnung}
\end{note}

\begin{note}
    \xplain{Optimieren Sie eine Mobiltelefontastatur: Nach der Nummerneingabe muss die Taste “Abheben” zum Wählen gedrückt werden. Wieviel bringt eine Verkürzung des mittleren Abstands zwischen dem Ziffernblock und „Abheben“ von 3cm auf 1cm bei einer Tastengröße von 5x5mm?}
    \xplain{Gute Frage.}
\end{note}

\begin{note}
    \xplain{Bei welchen Aufgaben bei der Gestaltung von grafischen Benutzungsoberflächen sollten Gestaltgesetze beachtet werden?}
    \xplain{idk}
\end{note}

\begin{note}
    \xplain{Welche Bedeutung hat der Kanal der taktilen Wahrnehmung in der Mensch-Computer-Interaktion?}
    \xplain{kk}
\end{note}

\begin{note}
    \xplain{Wie können in der Mensch-Computer-Interaktion Missverständnisse vermieden werden?}
    \xplain{kk}
\end{note}

\begin{note}
    \xplain{Nennen und begründen Sie einige Beispiele für Constraints und Affordances - gibt es kulturelle Unterschiede? Bei welchen Gegenständen würden sie sich Affordances oder Constraints wünschen?}
    \xplain{s}
\end{note}

\begin{note}
    \xplain{Welche Metapher würden Sie vorschlagen, um damit Leihwagen in einem Konzern zu verwalten?}
    \xplain{d}
\end{note}

\begin{note}
    \xplain{Erklären Sie den Unterschied zwischen Konsistenz und Erwartungskonformität.}
    \xplain{sd}
\end{note}

\begin{note}
    \xplain{Welche Kriterien wiederholen sich in verschiedenen Richtlinien? Worauf deutet diese Redundanz hin?}
    \xplain{sd}
\end{note}

\begin{note}
    \xplain{Ordnen Sie die Beispiele (Verteilung in Übungsstunde) den entsprechenden Dialogprinzipien für Interaktionsgestaltung zu.}
    \xplain{sdf}
\end{note}

\end{document}